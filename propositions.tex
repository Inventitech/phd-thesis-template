\documentclass{dissertation}

%% Turn off page numbering for the propositions and make the margins on both
%% sides equal and symmetrical.
\geometry{twoside=false}
\pagestyle{empty}

\begin{document}

%% Specify the title and author of the thesis. This information will be used on
%% both the English and Dutch side and in the metadata of the final PDF.
\title{An Empirical Evaluation of Feedback-Driven Development}
\author{Moritz Marc}{Beller}

\begin{center}

{\Large\titlefont\bfseries Propositions}

\bigskip

accompanying the dissertation

\bigskip

%% Print the title.
{\makeatletter
\titlestyle\bfseries\large\@title
\makeatother}

%% Print the optional subtitle.
{\makeatletter
\ifx\@subtitle\undefined\else
    \titlefont\titleshape\@subtitle
\fi
\makeatother}

\bigskip

by

\bigskip

%% Print the full name of the author.
\makeatletter
{\large\titlefont\bfseries\@firstname\ {\titleshape\@lastname}}
\makeatother

\end{center}

\bigskip
\bigskip

\begin{enumerate}
\item Researchers should study testing practices in a language whose ecosystem is testing-focused
  (such as Ruby) rather than Java, which seems to be the current default. [This thesis]
\item There is a dichotomy among OSS projects: Few projects are swarmed by contributions and do not
  know how to manage them anymore, whereas most barely get any, or if they do, they are not
  integrated because the project is already dormant. Paradoxically, both kinds of projects would
  benefit from automating their feedback loops. [This thesis]
\item Once we have uncovered or researched a new phenomenon empirically, we can never research it
  again from the same standpoint because we have interfered with it. [This thesis]
\item Most papers contain at least one statistical result that is wrong. [This thesis]
\item Most published empirical research only captures a static snapshot of current software
  development practices and is thus at best only of contemporary value.
\item The fundamental problem of Empirical Software Engineering, in contrast to most other science
  fields, is its inherent lack of generality in the outcome measures that would best describe
  different projects.
\item Reviewers who write comments on the lines of ``this finding is not interesting or
  surprising'' should read crime novels instead of scientific papers.
\item Most citations are superficial and do not embed or work with the cited paper.
  \newpage
\item[9a.] Reviewers are more likely to reveal their identity to authors in the case of a positive
  assessment, creating a dependency of gratitude on the authors. Because the Software Engineering
  research community is so interwoven, severe conflicts of interest arise from this behavior,
  intended or not.
\item[9b.] Everybody in academia has their fundamentally own agenda in joint work. In most roles,
  this means doing as little work as possible, while participating in as many projects as possible,
  is beneficial. Hence, most of the time, no real team work can exist in academia.
\item[9c.] With teaching, researching, and attracting funding, professors at universities have too
  many roles to fulfill them satisfactorily. Thus, by definition, professors can not do a good job.
\item[10.] The productivity of PhD students is directly correlated with how nice their supervisors
  are.
\end{enumerate}

\bigskip
\bigskip

%% Apart from the name and title of the supervisor, the following text is
%% dictated by the promotieregelement.
\begin{center}
These propositions are regarded as opposable and defendable, and have been approved as such by the
promotors prof.\ dr.\ A.\ van Deursen, dr.\ A.\ Zaidman, and dr.\ G.\ Gousios.
\end{center}

%% \clearpage
%% {\selectlanguage{dutch}

%% \begin{center}

%% {\Large\titlefont\bfseries Stellingen}

%% \bigskip

%% behorende bij het proefschrift

%% \bigskip

%% %% Print the title.
%% {\makeatletter
%% \titlestyle\bfseries\large\@title
%% \makeatother}

%% %% Print the optional subtitle.
%% {\makeatletter
%% \ifx\@subtitle\undefined\else
%%     \titlefont\titleshape\@subtitle
%% \fi
%% \makeatother}

%% \bigskip

%% door

%% \bigskip

%% %% Print the full name of the author.
%% \makeatletter
%% {\large\titlefont\bfseries\@firstname\ {\titleshape\@lastname}}
%% \makeatother

%% \end{center}

%% \bigskip
%% \bigskip

%% \begin{enumerate}

%% \item Stelling 1.
%% \item Stelling 2.
%% \item Stelling 3.
%% \item Stelling 4.
%% \item Stelling 5.
%% \item Stelling 6.
%% \item Stelling 7.
%% \item Stelling 8.
%% \item Stelling 9.
%% \item Stelling 10.

%% \end{enumerate}

%% \bigskip
%% \bigskip

%% %% Apart from the name and title of the supervisor, the following text is
%% %% dictated by the promotieregelement.
%% \begin{center}
%% Deze stellingen worden opponeerbaar en verdedigbaar geacht en zijn als zodanig goedgekeurd door de promotoren prof.\ dr.\ A.\ van Deursen and dr.\ A.\ Zaidman.
%% \end{center}

%% }

\end{document}

