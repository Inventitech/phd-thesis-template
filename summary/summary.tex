\chapter*{Summary}
\addcontentsline{toc}{chapter}{Summary}
\setheader{Summary}

Software developers today crave for feedback, be it from their peers in the form of code review,
static analysis tools like their compiler, or the local or remote execution of their tests in the
Continuous Integration (CI) environment. With the advent of social coding sites like \github and
tight integration of CI services like \travis, software development practices have fundamentally
changed.  Despite a highly alternated software engineering landscape, however, we still lack a
suitable holistic description of contemporary software development practices. Existing descriptions
like the V-model are either too coarse-grained to describe an individual contributor's workflow, or
only regard a sub-part of the development process like Test-Driven Development (TDD). In addition,
most existing models are \emph{pre-} rather than \emph{de-}scriptive.

By contrast, in our thesis, we perform a series of empirical studies to describe the individual
constituents of Feedback-Driven Development (FDD): we study the prevalence and evolution of
Automatic Static Analysis Tools (ASATs), we explain the ``Last Line Effect,'' a phenomenon at the
boundary between ASATs and code review, we observe local testing patterns in the IDE, compare them
to remote testing on the CI server, and, finally, should these quality assurance techniques have
failed, we examine how developers debug faults. We then compile this empirical evidence into an
initial theory of how modern software development works.

Our results show that developers employ the different techniques in FDD to best achieve their
current task in the most efficient way, often knowingly taking shortcuts to \emph{get the job
  done}. While this is efficient in the short term, it also bears risks, namely that prevention and
introspection activities fall short: developers might not configure or combine ASATs to their full
benefit, they might have wrong perceptions about the amount of time spent on quality-control, that
quality-related activities like testing become an after-thought, and learning about debugging
techniques falls short. A relatively rigid, tool-enforced FDD process could help developers in not
committing some of these fallacies. Our thesis culminates in the finding that feedback loops are
the characterizing criterion of contemporary software development. Our model is flexible enough to
accommodate a broad bandwidth of modern workflows, despite large variances in how projects use and
configure parts of FDD.

\chapter*{Samenvatting}
\addcontentsline{toc}{chapter}{Samenvatting}
\setheader{Samenvatting}

{\selectlanguage{dutch}

Samenvatting in het Nederlands\ldots

}

